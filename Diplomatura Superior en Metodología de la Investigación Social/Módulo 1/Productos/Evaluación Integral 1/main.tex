%%%%%%%%%%%%%%%%%%%%%%%%%%%%%%%%%%%%%%%%%
% Lachaise Assignment
% LaTeX Template
% Version 1.0 (26/6/2018)
%
% This template originates from:
% http://www.LaTeXTemplates.com
%
% Authors:
% Marion Lachaise & François Févotte
% Vel (vel@LaTeXTemplates.com)
%
% License:
% CC BY-NC-SA 3.0 (http://creativecommons.org/licenses/by-nc-sa/3.0/)
% 
%%%%%%%%%%%%%%%%%%%%%%%%%%%%%%%%%%%%%%%%%

%----------------------------------------------------------------------------------------
%	PACKAGES AND OTHER DOCUMENT CONFIGURATIONS
%----------------------------------------------------------------------------------------

\documentclass{article}

\input{structure.tex} % Include the file specifying the document structure and custom commands

%----------------------------------------------------------------------------------------
%	ASSIGNMENT INFORMATION
%----------------------------------------------------------------------------------------

\title{Metodologías-2025: Evaluación Integral \#1} % Title of the assignment

\author{Luis Fernando González Avila} % Author name and email address

\date{Consejo Latinoamericano de Ciencias Sociales --- \today} % University, school and/or department name(s) and a date

%----------------------------------------------------------------------------------------

\begin{document}

\maketitle % Print the title

%----------------------------------------------------------------------------------------
%	INTRODUCTION
%----------------------------------------------------------------------------------------

\section{Crítica a las nociones positivistas de verdad, a los modelos androcéntricos de hacer ciencia} % Unnumbered section

Los fundamentos epistemológicos, vistos desde una perspectiva moderna, mas no contemporánea, tal y como se han estado practicando, han conseguido que se creen críticas hacia los mismos, particularmente al positivismo con su noción de verdad (natural y absoluta) y correspondencia con los hechos de manera "objetiva", y que ponen en duda la neutralidad de la ciencia como objeto de estudio por todas las personas. Esta crítica tiene como uno de sus sustentos más empedernidos la ruptura con la idea "tradicional" de que para que exista un avance en las ciencias -revoluciones científicas- sólo es necesario acumular conocimientos en favor de que son las \textit{rupturas paradigmáticas}, culturales e históricamente situadas las que dan paso a que las ciencias sigan evolucionando.\\

\noindent Como bien se mencionaba, el objeto de estudio por todas las personas indica que, tanto el género masculino como el género femenino aportan al desarrollo del mismo; y es justamente el giro feminista el que puede aportar una dimensión ética y política a esta discusión epistemológica.

\begin{info} % Information block
	Norma Blázquez Graf (2012) plantea que las epistemologías feministas denuncian que la ciencia ha sido históricamente formulada y construida desde un punto de vista androcéntrico, y que esta construcción, con el paso de los siglos se ha normalizado a tal punto de que la idea de una ruptura con las formas "tradicionales" pudieran no ser vistas críticamente positivas, y que se \textit{invisibilizan} no sólo a las mujeres (o sus puntos de vista) sino también a otras subjetividades no hegemónicas.
\end{info}

\noindent Por lo que el conocimiento podría ser concebido como una producción que está ligada a los valores, creencias, costumbres y condiciones sociales y materiales de aquel sujeto que funge como investigador. Una alternativa propuesta es la de la \textit{objetividad fuerte} por Harding, en la que se busca una ciencia que esté consciente de los sesgos a los cuales escapa y de los contextos a los cuales pertenece para "combatir" la entredicha neutralidad del investigador moderno.

\section{Los problemas metodológicos presentes en el artículo “Cómo llegar a ser doctor en Sociología sin poseer el oficio de sociólogo”} % Numbered section

Bernard Lahire (2001) ejemplifica en el artículo propuesto "Cómo llegar a ser doctor en Sociología sin poseer el oficio de sociólogo" de Teissier cómo una práctica de investigación que aparenta ser académica, no llega a cumplir con aquellos criterios mínimos de cientificidad que son exigidos en el campo de la sociología, ya que identifica algunos problemas que son graves:

\begin{enumerate}
	\item No se define una sólida problematización con respecto a la teoría(s) propuestas; utiliza términos como sociedad de manera acrítica, en donde no reconoce ni su relevancia histórica ni su construcción social para dentro del problema planteado.
	\item Es dificultoso evaluar sus afirmaciones ya que no define claramente cual es la metodología implementada o cuál es la hipótesis última sobre la cual gira en torno la investigación.
	\item -Tal vez recuperando un poco sobre la clase 3 de la \textit{triangulación de metodologías}- Al no tener una claridad en las unidades de análisis utilizadas, es complicado delimitar cómo sustenta sus conclusiones y con respecto a qué objetivo cualitativo o cuantitativo está otorgando una respuesta.
\end{enumerate}

Pero también (y apelando a la reflexión anterior sobre el punto uno), la investigación de Teissier contiene afirmaciones que están basadas en intuiciones meramente personales, despojándola así de las pretensiones a pertenecer al campo académico riguroso; convirtiéndose así en una forma propia que la autora concibe para la legitimación de creación de conocimientos que no corresponden a la estandarización propia del campo sociológico.\\

\noindent Esto se traduce, tal y como lo menciona Lahire, en que para las ciencias -sociales- es indispensable tener una alineación metodológica adecuada y coherente durante todo el proceso de producción teórica, empírica y analítica.

%----------------------------------------------------------------------------------------

\end{document}
